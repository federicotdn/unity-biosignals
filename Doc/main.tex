\documentclass[a4paper]{report}
\usepackage[T1]{fontenc}
\usepackage[utf8]{inputenc}
\usepackage[spanish]{babel}
\usepackage{amsmath}
\usepackage{blkarray}
\selectlanguage{spanish}
\usepackage{url}
\usepackage[document]{ragged2e}
\usepackage{graphicx}
\usepackage{cite}
\usepackage{babelbib}
\usepackage{float}
\usepackage{appendix}
\usepackage[section,numberedsection=autolabel,acronym]{glossaries}
\graphicspath{ {images/} }

\title{
	{\LARGE Proyecto Final}\\
	{\Huge Procesamiento de Bioseñales en Tiempo Real en Universos Interactivos}\\
	{\large Instituto Tecnológico de Buenos Aires}\\~\\
	{\includegraphics{itba.png}}
}

\author{{Federico Tedin} \\ {Javier Fraire}}
\date{Febrero 2017}

\makeindex
\makeglossaries

\setglossarypreamble[acronym]{Se preservaron los acrónimo en inglés para facilitar la búsqueda de los mismos y para mantener consistencia con las publicaciones ya existentes. A continuación se presenta una lista de los acrónimos utilizados y su definición.}

\newacronym{psd}{PSD}{\emph{Power Spectral Density}}
\newacronym{spo2}{SpO\textsubscript{2}}{Pulsioximetría}
\newacronym{eeg}{EEG}{Electroencefalografía}
\newacronym{emg}{EMG}{Electromiografía}
\newacronym{acat}{ACAT}{\emph{Assistive context-aware toolkit}}
\newacronym{nasa}{NASA}{\emph{National Aeronautics and Space Administration}}
\newacronym{ekg}{EKG}{Electrocardiograma}
\newacronym{gsr}{GSR}{\emph{Galvanic Skin Response}}
\newacronym{fft}{FFT}{\emph{Fourier Fast Transform}}
\newacronym{svm}{SVM}{\emph{Support Vector Machines}}
\newacronym{lda}{LDA}{\emph{Linear Discriminant Analisis}}
\newacronym{bci}{BCI}{\emph{Brain-computer Interface}}
\newacronym{muap}{MUAP}{\emph{Motor Unit Action Potential}}
\newacronym{osc}{OSC}{\emph{Open Sound Control System}}
\newacronym{udp}{UDP}{\emph{User Data Protocol}}
\newacronym{sdk}{SDK}{\emph{Software Development Kit}}
\newacronym{cil}{CIL}{\emph{Common Intermediate Language}}
\newacronym{il2cpp}{IL2CPP}{\emph{Intermediate Language to C++}}
\newacronym{ecs}{ECS}{\emph{Entity-Component System}}
\newacronym{bpm}{BPM}{\emph{Beats per Minute}}

\begin{document}
\maketitle

\chapter*{Abstract}
\justifying
Biosignals are defined as signals produced by live tissues. The techniques used to obtain biosignals in this project were: \acrshort{eeg}, which measures brain bioelectrical activity, \acrshort{emg}, which measures electrical activity generated by the muscles, and \acrshort{spo2}, which measures oxygen levels in the blood.

After being read, the signals were processed in order to remove noise, and extract certain features of interest. Classifiers were used to determine if the resulting features represented certain actions taken by the user. These were: wether the user was with his or her eyes closed or open, and wether the user was tensing the muscles of his or her arm or not.

The data obtained was used as an input in various interactive worlds, in order to give the users a more inmersive experience. It was concluded that the use of biosignals in these interactive worlds is not only possible, but that they significantly improved the user's experience. Even then, there's still room for improvement, like increasing the classification process' precision, or finding different uses for the signals.

\chapter*{Resumen}
Las bioseñales consisten en señales producidas por los tejidos vivos. Las bioseñales se han utilizado en la medicina y en el campo de la accesibilidad. Su uso permite adquirir informacion sobre el estado del paciente que puede ser utilizada para diagnosticos, para asistir a pacientes con dificultades de accesibilidad, entre otros. Por otro lado Los Universos Interactivos son simulaciones visuales que reaccionan ante entradas del usuario, y han sufrido un auge en las últimas décadas debido al avance de los videojuegos, la realidad virtual y la realidad aumentada.  Tradicionalmente los mecanismos de interacción han sido el \emph{mouse}, teclados y más recientemente \emph{gamepads} y acelerómetros existentes en cascos de realidad virtual.  Este trabajo intenta responder si es posible implementar mecanismo de interacción basado en bioseñales para el desarrollo de Universos Interactivos.  Particularmente para esta ponencia, nos hemos concentrado en las bioseñales que permiten un balance entre practicidad, precio y disponibilidad, tiempo de procesamiento, y el uso que se le puede dar dentro de los universos interativos. Las técnicas utilizadas para obtener bioseñales en este trabajo han sido: \gls{eeg}, la cual consiste en la actividad bioeléctrica cerebral, \gls{emg}, la cual contempla la actividad eléctrica generada por los músculos del cuerpo, y \gls{spo2}, la cual permite determinar la oxigenación en sangre.

En base a esta información obtenida de las bioseñales, se la utilizó como entrada alternativa en diversos universos interactivos, dándole posiblemente una mayor inmersión a los usuarios. Para lograr este propósito, se realizó un análisis de las características de las señales y de cuáles eran los mecanismos de preprocesamiento (filtrado, reducción del ruido), extracción de características y finalmente clasificación para obtener indicadores que permitiesen interactuar con los Universos Interactivos.   Finalmente, se concluyó que el uso de bioseñales en tiempo real en estos universos no solo es posible, si no que mejoraron significativamente la experiencia del usuario. Aún así, existe lugar para mejoras, como aumentar la fidelidad de la clasificación de las señales, o encontrar distintas aplicaciones para las mismas.

Se utilizó la información obtenida como entrada en diversos universos interactivos, dándole una mayor inmersión a los usuarios. Finalmente, se concluyó que el uso de bioseñales en tiempo real en estos universos no solo es posible, si no que mejoraron significativamente la experiencia del usuario. Aún así, existe lugar para mejoras, como aumentar la fidelidad de la clasificación de las señales, o encontrar distintas aplicaciones para las mismas.

\tableofcontents

\chapter{Introducción}
Este trabajo tuvo como objetivo el procesamiento de bioseñales en tiempo real en universos interactivos. Una bioseñal puede ser definida como una descripción de un fenómeno fisiológico \cite{biosignal-book-2}.  El término ``tiempo real'', aquí, se refiere a computación gráfica en tiempo real, es decir, la generación de imágenes lo suficientemente rápido como para crear la ilusión de movimiento. Los universos interactivos son simulaciones que obtienen entradas del usuario y reaccionan antes las mismas. Las entradas más comunes son el ratón y el teclado. En este trabajo se investigó la utilización de bioseñales como entradas adicionales para aumentar la inmersión.

Se han encontrado muy pocas aplicaciones de uso comercial de bioseñales en universos interactivos y su alcance es muy reducido. Se buscó determinar si \acrshort{eeg}, \acrshort{emg} y \acrshort{spo2} eran aptos para la utilización en tiempo real en universos interactivos y cómo su utilización mejoraría la experiencia de usuario, contando sólo con dispositivos que puedan ser adquiridos en el mercado, y no dispositivos profesionales de alto costo. El objetivo de este proyecto no fue ofrecer un análisis cualitativo de los diferentes métodos de procesamiento de señales, sino cómo utilizar los existentes para mejorar la inmersión en universos interactivos teniendo en cuenta los desafíos que esto implica.

\chapter{Distintas Aplicaciones de Bioseñales} \label{chap:biosignal-apps}
En este capítulo se presentará el estado del arte en lo que respecta a los distintos usos de las bioseñales. Se brindarán algunos ejemplos del uso de las bioseñales en distintos campos.

\section{\acrshort{acat}}

La computadora utilizada por Stephen Hawking es tal vez el caso más conocido de la utilización de bioseñales en accesibilidad. Stephen Hawking cuenta con esclerosis lateral amiotrófica, por lo que se encuentra paralizado y no puede hablar. Para poder comunicarse, Intel desarrolló un sistema compuesto por una tableta y un sensor infrarojo montado sobre sus anteojos. El sensor infrarojo detecta el movimiento en su mejilla izquierda. La tableta cuenta con una plataforma de código abierto llamada \acrshort{acat}. \acrshort{acat} provee un teclado virtual en la pantalla. Utilizando el movimiento de su mejilla, Hawking, puede detener el cursor donde desea y así, escribir. Es decir, es una entrada binaria. Este también utiliza un procesador de texto con predicción de palabras que permite acelerar el proceso.  Luego, el sistema utiliza un sintetizador de voz para comunicar lo que escribió. Esta es tan solo una de las aplicaciones de \acrshort{acat}. \acrshort{acat} también le permite controlar el ratón en \emph{Windows}, y así, controlar completamente la computadora para poder utilizar su correo electrónico, navegar por internet, entre otras cosas. \acrshort{acat} puede utilizar como entrada cualquier bioseñal. \cite{hawking}.

\section{Gestos Como Dispositivos de Entrada}

La \acrshort{nasa} desarrolló un sistema para controlar un avión en una simulación utilizando los movimientos musculares medidos con sensores \acrshort{emg}. Colocaron diversos sensores sobre una manga de tela. Con ellos, adquirieron la señal y la filtraron y eliminaron el ruido. Luego extrajeron las características y reconocieron patrones en una fase de entrenamiento. Con esta información, se aplicaron patrones de reconocimiento en una simulación interactiva. Lograron controlar un avión de guerra sin utilizar una palanca de mando. Es decir, el usuario colocaba la mano como si estuviese utilizando una palanca de mando y realizaba movimientos para controlar el avión (ver figura \ref{fig:emg-flight}) \cite{emg-flight}.

\begin{figure}[H]
	\centering
    \includegraphics[width=0.8\textwidth]{emg-flight.png}
    \caption{Un usuario utilizando el dispotivo EMG para controlar un avión en una simulación  \cite{emg-flight}.}
	\label{fig:emg-flight}
\end{figure}

\section{\emph{Muscleman}}

Dos académicos de la Universidad Nacional de Seúl, utilizaron un dispositivo \acrshort{emg} y un acelerómetro para controlar un vídeojuego. Utilizando el acelerómetro, el juego era capaz de determinar si el usuario estaba dando un simple puñetazo hacia adelante, un puñetazo de abajo hacia arriba o si estaba lanzando una bola de fuego (ver figura \ref{fig:fireball}). Usando el sensor EMG, el juego medía la fuerza realizada por el usuario y la aplicaba proporcionalmente en el juego. Es decir, si el usuario realizaba poca fuerza, el ataque era débil. En cambio, si era fuerte, el ataque era fuerte. De esta forma, se utilizó como dispositivo de entrada las propias señales del cuerpo en lugar de usar un control de mando físico o el teclado \cite{emg-fireball}.

\begin{figure}[H]
    \includegraphics[width=0.8\textwidth]{emg-fireball.png}
    \caption{Movimiento que debe realiza un usuario para lanzar una bola de fuego en el videojuego \emph{Muscleman} \cite{emg-fireball}.}
	\label{fig:fireball}
\end{figure}

\section{\emph{Muse}}

La empresa \emph{Muse} desarrolló un dispositivo \acrshort{eeg} con siete electrodos. El mismo viene acompañado con una aplicación móvil que ayuda a los usuarios a meditar. Cuando el usuario tiene la mente tranquila, se escucha un clima calmo, pero cuando el usuario está alterado se escucha un clima tormentoso. Muse utiliza distintas ondas cerebrales para detectar si el usuario se encuentra relajado o no.

\section{\emph{MindFlix}}

\emph{Netflix} desarrolló \emph{MindFlix}. \emph{Mindflix} utiliza un dispositivo \acrshort{eeg} para controlar su popular servicio con la mente. Utiliza el giroscopio y el acelerómetro del dispositivo para permitirle al usuario desplazarse horizontalmente y verticalmente por la interfaz. Además, utiliza distintas ondas cerebrales para detectar cuando el usuario piensa en la palabra \emph{play}. En caso de que el usuario piense en esa palabra,  la aplicación comienza a reproducir el contenido seleccionado. Se intentó averiguar qué ondas cerebrales se utilizaban y de que forma, pero no se encontró en ningún lugar \cite{mindflix}.


\section{\emph{Neurogaming}}

\emph{Neurogaming} es la utilización de \acrshort{bci} en videojuegos para mejorar la experiencia. El concepto surgió hace algunos años pero aún no se ha explorado mucho. Existe muy pocas aplicaciones comerciales de este tipo. Un ejemplo es \emph{Throw Trucks With Your Mind} el cuál utiliza las ondas \emph{Beta} del cerebro para lanzar camiones \cite{neurogaming}.




\chapter{Marco Teórico}
Dado que este proyecto se centrará en las bioseñales, resulta fundamental explicar los conceptos necesarios para su entendimiento. Primero se hablará brevemente sobre el procesamiento de las señales. Luego se introducirán los conceptos de extracción de características y clasificación. Finalmente, se explicará como funcionan algunos sensores y se introducirá información teórica sobre la información que se obtiene de los mismos.

\section{Procesamiento de señales}

Las señales obtenidas de los sensores poseen ruido debido a que el \emph{hardware} no es 100\% confiable. Para eliminar el ruido se utilizan tantos filtros por \emph{hardware}, que los aplica el propio sensor, y filtros por \emph{software}. Dentro de los filtros de \emph{software} se encuentra el filtro \emph{Gaussiano}. Dicho filtro suaviza la señal por lo que elimina los picos que se pudieran originar por ruido propio del sensor. Además, el filtro \emph{Gaussiano}, a diferencia de otros filtros, no elimina las altas frecuencias completamente. El filtro \emph{Gaussiano} se aplica haciendo una convolución de la señal con la siguiente función:

$$ g(x) = \frac{1}{\sqrt{2 * \pi} * \sigma } * e^{-\frac{x^{2}}{2 * \sigma^{2}}} $$

Una vez que se redujo el ruido, se pueden aplicar otros filtros o utilizarla directamente. Muchos sensores tienen como salida el nivel de potencial eléctrico medidas en $\mu V$ (microvoltios). Esta información sin ningún tipo de procesamiento no es útil. Dependiendo de que se quiera detectar se pueden realizar distintas operaciones. Una de ellas, es la búsqueda de picos. La primera derivada de un pico tiene un cruce descendente igual a cero en su máximo. Por ello, lo que se hace es primero suavizar la señal para eliminar ruido y luego se calculan las derivadas cruzadas. Luego, si la pendiente excede un umbral, significa que se ha encontrado un cero \cite{peak-finding}. Estos picos encontrados representan distintas cosas dependiendo el sensor utilizado. Por ejemplo, al utilizar un EMG, puede significar un impuslo de fuerza. En un EEG, puede significar un pestañeo.

Otro procesamiento que se le puede aplicar es la transformada discreta de \emph{Fourier}. La transformada de \emph{Fourier} transforma una función que se encuentra en el dominio del tiempo a una función que se encuentra en el dominio de la frecuencia. La transformada de \emph{Fourier} se define de la siguiente manera:

$$ x_{k} = \sum_{n=0}^{N-1} x_{n}e^{-\frac{2 \pi i}{N}kn} \qquad k = 0,..., N - 1 $$

Una vez que la función se encuentra en el dominio de la frecuencia, se puede proceder con el procesamiento. Se selecciona el rango de frecuencias de interés y se le aplica un filtro pasa banda, que deja pasar un determinado rango de frecuencias de una señal y atenúa el resto. Luego de aplica el filtro pasa banda,  se cuenta con las frecuencias de interés y se continuá con el procesamiento. Una alternativa es calcular la Densidad Espectral de Potencia (DEP). Esta se define como:

$$ P = \int_{-\inf}^{+inf} S_{xx} (f) df \qquad  \textrm{donde}$$

$$ S_{xx} = |X(f)|^{2} \qquad \textbf{y} \qquad X(f) \textrm{ es la Transformada de \emph{Fourier}} $$

Esta potencia puede ser utilizada luego como una característica de interés. Se discutirá más adelante. Otra alternativa  es, por ejemplo, calcular el promedio de las frecuencias. Las posibilidades aquí son muchas y dependen de lo que se esté buscando.

\section{Extracción de Características de Interés y Clasificación}

\section{EEG}

\section{EMG}

\section{EKG}

\chapter{Implementación}
\section{Hardware}

\section{Procesamiento de Señales}

\section{Desarrollo de universos 3D interactivos}

\chapter{Resultados y Análisis}
En este capítulo se presentan los resultados obtenidos al procesar las señales y se realiza un breve análisis. Al registrar sesiones,  se creaba una única sesión y luego se dividía en múltiples partes. Debido a que el posicionamiento del sensor es importante y varía cuando se coloca y se retira,  y a que el estado del cuerpo cambia de un instante al otro, las bioseñales sufren variaciones. Por este motivo, se utilizaban datos de una única sesión tanto como para entrenar como para predecir. De esta manera, se buscaba que la señal sea lo más consistente posible. Para calcular la precisión se utilizó el método de validación cruzada de $k$ iteraciones. Se tomó esta decisión ya que, como se mencionó en la sección \ref{sec:classification}, partir las sesiones en dos partes y utilizar una para entrenamiento y otra para validación no resulta representativo debido a que la elección del lugar en donde se parten los datos puede influir mucho en el resultado. Se utilizaron valores de $k$ iguales a $5$, $10$ y $15$. La elección de estos valores se basó en la cantidad de muestras con la que se contaba. Luego, se promediaban los resultados arrojados por cada valor de $k$ y ese número reflejaba la precisión de la sesión. Tanto en EEG como en EMG se analizaron los siguientes clasificadores: \emph{naive Bayes},  \acrshort{lda}, \gls{svm} y árboles de decisión. En ningún caso se realizó un análisis del por qué algunos clasificadores resultaron mejores que otros ya que no fue el foco de este proyecto.

\section{\acrshort{eeg}}

Como se mencionó en la sección \ref{sec:eeg-signal-processing} del capítulo anterior, primero se comenzó utilizando los valores de potencia \emph{Alfa} que brindaba el sensor ($\alpha_{10 \, Hz}$). Para intentar mejorar la precisión, se decidió realizar una implementación propia ($\alpha_{256 \, Hz}$) y confeccionar el vector de características con cuatro elementos. Para determinar que clasificador utilizar, se utilizaron todos los conjuntos de datos y se promedio la precisión de cada clasificador. En la tabla \ref{tab:eeg-class-result} se pueden observar los resultados de cada clasificador.
 
\begin{table}[H]
\centering
\begin{tabular}{ |c|c|c| } 
 \hline
 Clasificador & Precisión $\alpha_{10 \, Hz}$ &  Precisión $\alpha_{256 \, Hz}$ \\ 
 \hline
 Bayesiano ingenuo & $0.732$  & $0.680$  \\
 \hline
 \gls{lda}  & $0.726$ & $0.767$ \\
  \hline
  \gls{svm} & $0.706$ & $0.542$ \\
  \hline
 Árbol de decisión & $0.723$ & $0.834$ \\
 \hline
\end{tabular}
\caption{Resultados de utilizar distintos clasificadores sobre todas las muestras.}
\label{tab:eeg-class-result}
\end{table}

Se puede observar que los mejores resultados se obtuvieron utilizando un árbol de decisión con un vector de cuatro características obtenido de $256$ muestras. Al de utilizar el método con un vector de dos características obtenidos a $10 \, Hz$, no hubo grandes diferencias en cuanto al clasificador por lo que se decidió utilizar un árbol de decisión para ambos métodos.

En la tabla \ref{tab:eeg-results} se pueden observar los resultados de todas las sesiones.

\begin{table}[H]
\centering
\begin{tabular}{ |c|c|c| } 
 \hline
 Sujeto & Precisión $\alpha_{10 \, Hz}$ &  Precisión $\alpha_{256 \, Hz}$ \\ 
 \hline
 1 & $0.565$ & $0.665$ \\
 \hline
 2 & $0.831$ & $0.881$ \\
 \hline
 3 & $0.768$ & $0.88125$ \\
 \hline
4  & $0.727$ & $0.909$ \\
 \hline



 \hline
\end{tabular}
\caption{Resultados de utilizar distintas formas de calcular potencia de \emph{Alfa} en distintos sujetos utilizando como clasificador un árbol de decisión.}
\label{tab:eeg-results}
\end{table}

La primer observación que se puede realizar es que utilizar $\alpha_{256 \, Hz}$ es más preciso que utilizar $\alpha_{10 \, Hz}$. Como se mencionó anteriormente, esto se debe a que en el primer caso se toman $256$ muestras para calcular la potencia de \emph{Alfa} mientras que en el segundo se utilizan únicamente $25$. A su vez, en el primer caso se utilizan los valores obtenidos por cada electrodo por separado mientras que en el segundo caso se promedian y se arma el vector de características con dos valores consecutivos. Utilizar las mediciones de los distintos electrodos en un período de tiempo describe mejor el estado que utilizar valores de dos períodos consecutivos.

En las figuras \ref{fig:subject-2-10} y \ref{fig:subject-2-256} se observan gráficos de dispersión utilizando los valores de potencia de \emph{Alfa}. En ambos se observa una clara separación de estados. El estado de ojos cerrados contiene valores mayores que el estado de ojos abiertos. La figura \ref{fig:subject-2-10} cuenta con una cantidad mayor de valores atípicos, particularmente, una cantidad mayor de valores de \emph{Alfa} elevados en el estado de ojos abiertos. A su vez, en la figura \ref{fig:subject-2-256} los valores para ojos abiertos se encuentran mayormente concentrados por debajo de los valores para ojos cerrados. Por estos motivos, la precisión es mayor al utilizar el vector de cuatro dimensiones. Debido a la separación que se observa en los gráficos, fue suficiente utilizar un clasificador simple.

 \begin{figure}[H]
	\centering
    \includegraphics[width=0.8\textwidth]{subject-2-10.png}
    \caption{Gráfico de dispersión del vector de características de dos características del sujeto 2}
	\label{fig:subject-2-10}
\end{figure}

 \begin{figure}[H]
	\centering
    \includegraphics[width=0.8\textwidth]{subject-2-256.png}
    \caption{Gráfico de dispersión del vector de características de cuatro características del sujeto 2. Para transformar de cuatro dimensiones a dos, se promediaron los dos primeros valores y los dos valores finales.}
	\label{fig:subject-2-256}
\end{figure}


Al comparar las figuras \ref{fig:subject-1-256} y \ref{fig:subject-2-256}, se puede observar, que hay una mayor separación entre estados en la figura \ref{fig:subject-2-256}. Por este motivo, la precisión del sujeto $2$ fue un $20\%$ superior.

 \begin{figure}[H]
	\centering
    \includegraphics[width=0.8\textwidth]{subject-1-256.png}
    \caption{Gráfico de dispersión del vector de características de cuatro características del sujeto 1.}
	\label{fig:subject-1-256}
\end{figure}

Cabe destacar que al contar con un vector de cuatro características, para graficar se necesitarían cuatro dimensiones. Para abordar este problema se decidió transformar el vector en un vector de dos componentes donde la primer componente es el promedio de los dos primeros valores y la segunda, el promedio de los dos últimos valores.

Otro aspecto a analizar es el hecho de que la precisión varía mucho de sujeto a sujeto. Esto puede deberse a diversos motivos. Uno de ellos es que la persona al saber que está siendo analizada, se encuentra nerviosa por lo que puede afectar las ondas cerebrales. A su vez, si la persona se encuentra cansada, por ejemplo, los estímulos al campo visual tienden a ser menores, lo que genera que lo valores de \emph{Alfa} sean mayores. Esta gran diferencia puede apreciarse al observar que el Sujeto $1$ obtuvo una precisión mucho menor que el resto de los sujetos.

Si bien las ondas cerebrales varían mucho de persona a persona y de acuerdo al instante del tiempo, se realizó la prueba de utilizar como sesión de entrenamiento una sesión del sujeto 2  y como sesión de predicción una sesión del sujeto 1. Los resultados fueron sorprendentes ya que se obtuvo una precisión de $0.6537$. Este valor no es comparable con los obtenidos anteriormente ya que dichos valores utilizaron validación cruzada y éste simplemente utilizo un conjunto de datos para entrenamiento y otro para validación. De esta forma, se observa que el sistema de clasificación ofrece un determinado nivel de robustez.

\section{\acrshort{emg}}

Como se mencionó en la sección \ref{sec:emg-signal-processing}, los valores leídos fueron procesados para obtener un vector de dos características cada $128$ muestras, y luego fueron alimentados a un clasificador. Para la elección de clasificador, se realizaron pruebas sobre cuatro clasificadores distintos. Se tomaron seis sesiones de \acrshort{emg} grabadas, y se aplico la técnica de validación cruzada con $k$ iteraciones ( mencionada en la sección \ref{sec:classification}) por cada clasificador. Los resultados fueron los siguientes:

\begin{table}[H]
\centering
\begin{tabular}{ |c|c| }
 \hline
 Clasificador & Precisión \\ 
 \hline
 \emph{Naive Bayes} & $0.962$ \\
 \hline
 \gls{lda} & $0.951$ \\
 \hline
  \gls{svm} & $0.457$ \\
 \hline
 Árbol de decisión & $0.973$ \\
 
 \hline
\end{tabular}
\caption{Precisiones de distintos clasificadores sobre las sesiones de EMG.}
\label{tab:emg-classifiers}
\end{table}

Como se puede observar en la tabla \ref{tab:emg-classifiers}, los niveles de precisión de todos los clasificadores, excepto el de \acrshort{svm}, fueron muy elevados. Se decidió proceder a utilizar el de Árboles de decisión ya que produjo el resultado más elevado. No se realizaron pruebas adicionales ya que se consideró que el nivel de precisión alcanzado era lo suficientemente alto para fines prácticos de la simulación. No se utilizó un clasificador más complejo, como una red neuronal, ya que era posible resolverlo con clasificadores simples. Como se observará más adelante, existe una clara separación entre los valores de un estado y el otro por lo que un clasificador simple es más que suficiente. Además, los clasificadores más complejos requieren más poder de procesamiento y requieren una gran dedicación de tiempo para determinar los mejores parámetros. Luego de haber elegido el clasificador, se pudo realizar un estudio mas detallado, sesión por sesión. En la siguiente tabla se detalla: el sujeto utilizado, la precision alcanzada luego de entrenar al clasificador, y la duración total en segundos de las muestras tomadas. Para calcular la precisión obtenida en cada sesión, se utilizó nuevamente el método de validación cruzada de $k$ iteraciones.

\begin{table}[H]
\centering
\begin{tabular}{ |c|c|c| } 
 \hline
 Sujeto & Precisión & Duración (Segundos) \\ 
 \hline
 1 & $0.970$ & $603$ \\
  \hline
 2 & $0.955$ & $239$ \\
  \hline
 3 & $0.994$ & $393$ \\

 \hline
\end{tabular}
\caption{Resultados de entrenamientos utilizando lecturas de \acrshort{emg} para distintos sujetos. Frecuencia de muestreo: $256\,Hz$.}
\label{tab:emg-results}
\end{table}

Como muestran los valores de la tabla \ref{tab:emg-results}, los niveles de precisión alcanzados son muy elevados: en las tres sesiones realizadas, se logró un nivel de precisión de $95\%$, o mayor. Es necesario remarcar que al medir señales de \acrshort{emg}, existen varias fuentes de ruido eléctrico, que pueden afectar los valores leídos y por lo tanto perjudicar la precision del entrenador. El efecto neto de éstas fuentes de ruido varía de sesión en sesión, ya que depende de factores como la cantidad de movimiento de los electrodos en relación a la piel, o la presencia de dispositivos que emitan radiación electromagnética\cite{emg-delsys}.

A continuación, se muestran los gráficos de dispersión de las dos sesiones más largas, que permiten visualizar las diferencias entre los vectores de características de ambos estados de tensión de músculo.

 \begin{figure}[H]
	\centering
    \includegraphics[width=1\textwidth]{fede-1.png}
    \caption{Gráfico de dispersión del vector de características \acrshort{emg} para la sesión del sujeto $1$.}
	\label{fig:emg-graph-s1}
\end{figure}

 \begin{figure}[H]
	\centering
    \includegraphics[width=1\textwidth]{javo-1.png}
    \caption{Gráfico de dispersión del vector de características \acrshort{emg} para la sesión del sujeto $3$.}
	\label{fig:emg-graph-s3}
\end{figure}

En la figura \ref{fig:emg-graph-s1}, se puede observar una clara separación entre las muestras tomadas con el musculo relajado, y las muestras tomadas con el musculo tensado. En la figura \ref{fig:emg-graph-s3} se puede observar el mismo efecto, con una separación aun mas marcada. Esta clara separación fue la que permitió utilizar un clasificador simple. También, se puede notar como las muestras tomadas con músculo relajado se encuentran concentradas en un área pequeña, debido a la poca variación de potencial medido de muestra en muestra. Por el otro lado, las muestras tomadas con músculo tensado se encuentran dispersas, ya que al tensar el músculo no siempre es posible mantener el mismo nivel fuerza ejercida.

Como se menciono en la sección \ref{sec:emg-signal-processing}, la frecuencia de muestreo del modulo \acrshort{emg} fue cambiada de $256\,Hz$ a $512\,Hz$. Para comprobar que la precision de predicción se mantuvo luego de este cambio, se grabaron nuevas sesiones.

\begin{table}[H]
\centering
\begin{tabular}{ |c|c|c| } 
 \hline
 Sujeto & Precisión & Duración (Segundos) \\ 
 \hline
 1 & $0.972$ & $284$ \\
 \hline
 2 & $0.991$ & $339$ \\
 \hline
 3 & $0.984$ & $292$ \\
 \hline
 4 & $0.953$ & $130$ \\
 \hline
 5 & $0.969$ & $146$ \\

 \hline
\end{tabular}
\caption{Resultados de entrenamientos utilizando lecturas de \acrshort{emg} para varios sujetos. Frecuencia de muestreo: $512\,Hz$.}
\label{tab:emg-results-512}
\end{table}
	
Como se puede observar en la tabla \ref{tab:emg-results-512}, el nivel de precision se mantuvo elevado, luego de haber realizado el cambio de frecuencia de muestreo. Las dos sesiones mas largas fueron graficadas, nuevamente, para poder observar la relación entre los valores de los vectores de características y el estado del músculo:

\begin{figure}[H]
	\centering
    \includegraphics[width=1\textwidth]{fede-512-1.png}
    \caption{Gráfico de dispersión del vector de características EMG para la sesión del sujeto $1$ ($512\,Hz$).}
	\label{fig:emg-graph-s1-512}
\end{figure}

\begin{figure}[H]
	\centering
    \includegraphics[width=1\textwidth]{javo-512-1.png}
    \caption{Gráfico de dispersión del vector de características EMG para la sesión del sujeto $3$ ($512\,Hz$).}
	\label{fig:emg-graph-s3-512}
\end{figure}

Las figuras \ref{fig:emg-graph-s1-512} y \ref{fig:emg-graph-s3-512} muestran, nuevamente, una clara separación entre las muestras tomadas durante el tiempo que se mantuvo el músculo relajado y tensado.

En definitiva, los niveles de precision alcanzados utilizando las técnicas detalladas en este informe fueron muy elevados ($95\%$ o más en todos los casos). Éstos valores resultaron ser más que suficientes para el desarrollo de las simulaciones interactivas, que en sí también fueron diseñadas teniendo en cuenta que la precision de predicción prácticamente nunca podría ser del $100\%$.

\section{\acrshort{spo2}}

Para medir la precision del método explicado en la sección \ref{sec:spo2-signal-processing}, se grabó una sesión de aproximadamente un minuto para cuatro sujetos, y se comparó el promedio de \acrshort{bpm} obtenido durante la duración del intervalo.

\begin{table}[H]
\centering
\begin{tabular}{ |c|c|c| } 
 \hline
 Sujeto & Promedio de \acrshort{bpm} (Sensor) & Promedio de \acrshort{bpm} (Calculado) \\ 
 \hline
 1 & $78.380$ & $77.055$ \\
 \hline
 2 & $84.300$ & $80.775$ \\
 \hline
 3 & $74.953$ & $73.848$ \\
 \hline
 4 & $80.412$ & $77.004$ \\

 \hline
\end{tabular}
\caption{Promedios de \acrshort{bpm} para las sesiones de \acrshort{spo2} medidas.}
\label{tab:spo2-results}
\end{table}

Como se puede ver en la tabla \ref{tab:spo2-results}, los resultados obtenidos a través de nuestro método de calculo de \acrshort{bpm} son similares a los calculados internamente por el sensor. La diferencia mas grande entre ambos resultados, para todas las sesiones, fue de aproximadamente $4\%$. Ambos métodos para calcular \acrshort{bpm} tienen sus ventajas y desventajas: el que utiliza el sensor produce valores que fluctúan menos en el tiempo, pero representan un valor promediado sobre un intervalo de tiempo extenso, por lo que no responden rápidamente a cambios en el ritmo cardíaco. Por el otro lado, nuestro método responde mas rápidamente a los cambios, pero tiende a producir resultados que fluctúan con mayores magnitudes. Se graficaron los resultados de las sesiones de los sujetos $2$ y $3$ para poder comparar visualmente los resultados:

\begin{figure}[H]
	\centering
    \includegraphics[width=0.8\textwidth]{javier-1-spo2.png}
    \caption{Gráfico de \acrshort{bpm} a través del tiempo para la sesión del sujeto $2$.}
	\label{fig:spo2-graph-2}
\end{figure}

\begin{figure}[H]
	\centering
    \includegraphics[width=0.8\textwidth]{gabriela-1-spo2.png}
    \caption{Gráfico de \acrshort{bpm} a través del tiempo para la sesión del sujeto $3$.}
	\label{fig:spo2-graph-3}
\end{figure}

El gráfico \ref{fig:spo2-graph-2} muestra como ambos métodos de medición tienden a producir los mismos resultados, pero con algunas diferencias. El método del sensor tiende a producir los mismos valores producidos por nuestro método, pero cierto atraso. A la vez, nuestro método tiende a producir resultados que varían mas rápidamente con el tiempo. El gráfico \ref{fig:spo2-graph-3} muestra una sesión donde ambos métodos produjeron resultados mucho mas similares. En el caso de éste proyecto, se prefirió nuestro método, ya que la simulación interactiva requiere de un tiempo de respuesta lo mas corto posible a los cambios de ritmo cardíaco del usuario. Es necesario remarcar que aunque las mediciones de \acrshort{bpm} son discretas, se utilizaron gráficos de línea ya que éstos permitían interpretar la información más fácilmente. Los valores leídos están representados como puntos sobre las líneas.




\chapter{Conclusiones}
En este capítulo, se presentaran breves conclusiones sobre el proyecto realizado. 

En el caso de EEG se logró darle un uso que mejora la interacción del usuario con el universo. El problema de yace en que como las ondas \emph{Alfa} varían tanto de persona a persona, hay en personas que no funciona tan bien. Además, la precisión alcanzada, si bien es suficiente para el uso que se le dio, si se le quieren dar otros usos se debe mejorar. A su vez, el usuario debe quedarse muy quieto ya que los movimientos pueden causar que los electrodos generen lecturas incorrectas. Por este motivo, se concluye que la utilización de un EEG en un universo interactivo es realizable pero se deben obtener sensores que sean menos sensibles al movimiento.

El uso del dispositivo EMG fue sin dudas el que mejores resultados arrojó. Es muy fácil de colocar y su precisión fue muy alta. Además, con tan solo unos minutos de entrenamiento ya es suficiente para obtener buenas predicciones.

En el caso del dispositivo SpO\textsubscript{2}, es totalmente utilizable en tiempo real ya que la latencia es mínima y no requiere entrenamiento. El usuario simplemente se lo coloca y puede interactuar con el universo.

Por lo tanto, se concluye que posible procesar bioseñales en tiempo real y utilizar la información en universos interactivos. La pregunta difícil aquí es ¿cómo hacerlo?. Se debe hacer un análisis exhaustivo del estado que se quiera detectar para determinar que nivel mínimo de precisión se requiere y cuál sería el mejor uso dentro del universo. Además, el procesamiento debe ser lo suficientemente rápido para reaccionar rápidamente ante los cambios y no introducir latencia en el universo interactivo.

\appendix
\chapter{Apéndice}

\printglossary[type=\acronymtype,title={Lista de Acrónimos}]

\addcontentsline{toc}{chapter}{Bibliografía}
\bibliography{bibliography}{}
\bibliographystyle{babplain}

\end{document}