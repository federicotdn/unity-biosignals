\documentclass[a4paper]{report}
\usepackage[T1]{fontenc}
\usepackage[utf8]{inputenc}
\usepackage[spanish]{babel}
\usepackage{amsmath}
\usepackage{blkarray}
\selectlanguage{spanish}
\usepackage{url}
\usepackage[document]{ragged2e}
\usepackage{graphicx}
\usepackage{cite}
\usepackage{babelbib}
\usepackage{float}
\graphicspath{ {images/} }

\title{
	{\Huge Proyecto Final}\\
	{\Huge Interpretación de bioseñales para uso en aplicaciones interactivas}
	{\large Instituto Tecnológico de Buenos Aires}\\~\\
	{\includegraphics{itba.png}}
}

\author{{Federico Tedin} \\ {Javier Fraire}}
\date{Febrero 2017}

\makeindex

\begin{document}
\maketitle

\chapter*{Resumen}
\justifying
Las bioseñales consisten en señales producidas por tejidos vivos. Algunos ejemplos de bioseñales, en seres humanos, son: EEG (Electroencefalografía) la cual consiste en la actividad bioeléctrica cerebral, EMG (Electromiografía), la cual contempla la actividad eléctrica generada por los músculos del cuerpo, y ECG (Electrocardiograma), la cual representa la actividad eléctrica del corazón.

Dichas señales, luego de ser leídas, fueron procesadas para eliminar ruido y para extraer las características de interés. Con las características extraídas se utilizaron clasificadores para determinar si el usuario realizó una determinada acción. En este caso, se utilizó para deteterminar si el usuario estaba con los ojos abiertos o cerrados, y para determinar si el usuario estaba haciendo fuerza con la mano o no.

Con la información obtenida de los clasificadores se realizaron alteraciones en universos 3D dandolé una mayor inmersión a los usuarios.


\tableofcontents

\chapter{Estado del Arte}
En este capítulo se presentará el estado del arte en lo que respecta a los distintos usos de las bioseñales. Se brindarán algunos ejemplos del uso de las bioseñales en distintos campos.

\section{\acrshort{acat}}

La computadora utilizada por Stephen Hawking es tal vez el caso más conocido de la utilización de bioseñales en accesibilidad. Stephen Hawking cuenta con esclerosis lateral amiotrófica, por lo que se encuentra paralizado y no puede hablar. Para poder comunicarse, Intel desarrolló un sistema compuesto por una tableta y un sensor infrarojo montado sobre sus anteojos. El sensor infrarojo detecta el movimiento en su mejilla izquierda. La tableta cuenta con una plataforma de código abierto llamada \acrshort{acat}. \acrshort{acat} provee un teclado virtual en la pantalla. Utilizando el movimiento de su mejilla, Hawking, puede detener el cursor donde desea y así, escribir. Es decir, es una entrada binaria. Este también utiliza un procesador de texto con predicción de palabras que permite acelerar el proceso.  Luego, el sistema utiliza un sintetizador de voz para comunicar lo que escribió. Esta es tan solo una de las aplicaciones de \acrshort{acat}. \acrshort{acat} también le permite controlar el ratón en \emph{Windows}, y así, controlar completamente la computadora para poder utilizar su correo electrónico, navegar por internet, entre otras cosas. \acrshort{acat} puede utilizar como entrada cualquier bioseñal. \cite{hawking}.

\section{Gestos Como Dispositivos de Entrada}

La \acrshort{nasa} desarrolló un sistema para controlar un avión en una simulación utilizando los movimientos musculares medidos con sensores \acrshort{emg}. Colocaron diversos sensores sobre una manga de tela. Con ellos, adquirieron la señal y la filtraron y eliminaron el ruido. Luego extrajeron las características y reconocieron patrones en una fase de entrenamiento. Con esta información, se aplicaron patrones de reconocimiento en una simulación interactiva. Lograron controlar un avión de guerra sin utilizar una palanca de mando. Es decir, el usuario colocaba la mano como si estuviese utilizando una palanca de mando y realizaba movimientos para controlar el avión (ver figura \ref{fig:emg-flight}) \cite{emg-flight}.

\begin{figure}[H]
	\centering
    \includegraphics[width=0.8\textwidth]{emg-flight.png}
    \caption{Un usuario utilizando el dispotivo EMG para controlar un avión en una simulación  \cite{emg-flight}.}
	\label{fig:emg-flight}
\end{figure}

\section{\emph{Muscleman}}

Dos académicos de la Universidad Nacional de Seúl, utilizaron un dispositivo \acrshort{emg} y un acelerómetro para controlar un vídeojuego. Utilizando el acelerómetro, el juego era capaz de determinar si el usuario estaba dando un simple puñetazo hacia adelante, un puñetazo de abajo hacia arriba o si estaba lanzando una bola de fuego (ver figura \ref{fig:fireball}). Usando el sensor EMG, el juego medía la fuerza realizada por el usuario y la aplicaba proporcionalmente en el juego. Es decir, si el usuario realizaba poca fuerza, el ataque era débil. En cambio, si era fuerte, el ataque era fuerte. De esta forma, se utilizó como dispositivo de entrada las propias señales del cuerpo en lugar de usar un control de mando físico o el teclado \cite{emg-fireball}.

\begin{figure}[H]
    \includegraphics[width=0.8\textwidth]{emg-fireball.png}
    \caption{Movimiento que debe realiza un usuario para lanzar una bola de fuego en el videojuego \emph{Muscleman} \cite{emg-fireball}.}
	\label{fig:fireball}
\end{figure}

\section{\emph{Muse}}

La empresa \emph{Muse} desarrolló un dispositivo \acrshort{eeg} con siete electrodos. El mismo viene acompañado con una aplicación móvil que ayuda a los usuarios a meditar. Cuando el usuario tiene la mente tranquila, se escucha un clima calmo, pero cuando el usuario está alterado se escucha un clima tormentoso. Muse utiliza distintas ondas cerebrales para detectar si el usuario se encuentra relajado o no.

\section{\emph{MindFlix}}

\emph{Netflix} desarrolló \emph{MindFlix}. \emph{Mindflix} utiliza un dispositivo \acrshort{eeg} para controlar su popular servicio con la mente. Utiliza el giroscopio y el acelerómetro del dispositivo para permitirle al usuario desplazarse horizontalmente y verticalmente por la interfaz. Además, utiliza distintas ondas cerebrales para detectar cuando el usuario piensa en la palabra \emph{play}. En caso de que el usuario piense en esa palabra,  la aplicación comienza a reproducir el contenido seleccionado. Se intentó averiguar qué ondas cerebrales se utilizaban y de que forma, pero no se encontró en ningún lugar \cite{mindflix}.


\section{\emph{Neurogaming}}

\emph{Neurogaming} es la utilización de \acrshort{bci} en videojuegos para mejorar la experiencia. El concepto surgió hace algunos años pero aún no se ha explorado mucho. Existe muy pocas aplicaciones comerciales de este tipo. Un ejemplo es \emph{Throw Trucks With Your Mind} el cuál utiliza las ondas \emph{Beta} del cerebro para lanzar camiones \cite{neurogaming}.




\chapter{Marco Teórico}
Dado que este proyecto se centrará en las bioseñales, resulta fundamental explicar los conceptos necesarios para su entendimiento. Primero se hablará brevemente sobre el procesamiento de las señales. Luego se introducirán los conceptos de extracción de características y clasificación. Finalmente, se explicará como funcionan algunos sensores y se introducirá información teórica sobre la información que se obtiene de los mismos.

\section{Procesamiento de señales}

Las señales obtenidas de los sensores poseen ruido debido a que el \emph{hardware} no es 100\% confiable. Para eliminar el ruido se utilizan tantos filtros por \emph{hardware}, que los aplica el propio sensor, y filtros por \emph{software}. Dentro de los filtros de \emph{software} se encuentra el filtro \emph{Gaussiano}. Dicho filtro suaviza la señal por lo que elimina los picos que se pudieran originar por ruido propio del sensor. Además, el filtro \emph{Gaussiano}, a diferencia de otros filtros, no elimina las altas frecuencias completamente. El filtro \emph{Gaussiano} se aplica haciendo una convolución de la señal con la siguiente función:

$$ g(x) = \frac{1}{\sqrt{2 * \pi} * \sigma } * e^{-\frac{x^{2}}{2 * \sigma^{2}}} $$

Una vez que se redujo el ruido, se pueden aplicar otros filtros o utilizarla directamente. Muchos sensores tienen como salida el nivel de potencial eléctrico medidas en $\mu V$ (microvoltios). Esta información sin ningún tipo de procesamiento no es útil. Dependiendo de que se quiera detectar se pueden realizar distintas operaciones. Una de ellas, es la búsqueda de picos. La primera derivada de un pico tiene un cruce descendente igual a cero en su máximo. Por ello, lo que se hace es primero suavizar la señal para eliminar ruido y luego se calculan las derivadas cruzadas. Luego, si la pendiente excede un umbral, significa que se ha encontrado un cero \cite{peak-finding}. Estos picos encontrados representan distintas cosas dependiendo el sensor utilizado. Por ejemplo, al utilizar un EMG, puede significar un impuslo de fuerza. En un EEG, puede significar un pestañeo.

Otro procesamiento que se le puede aplicar es la transformada discreta de \emph{Fourier}. La transformada de \emph{Fourier} transforma una función que se encuentra en el dominio del tiempo a una función que se encuentra en el dominio de la frecuencia. La transformada de \emph{Fourier} se define de la siguiente manera:

$$ x_{k} = \sum_{n=0}^{N-1} x_{n}e^{-\frac{2 \pi i}{N}kn} \qquad k = 0,..., N - 1 $$

Una vez que la función se encuentra en el dominio de la frecuencia, se puede proceder con el procesamiento. Se selecciona el rango de frecuencias de interés y se le aplica un filtro pasa banda, que deja pasar un determinado rango de frecuencias de una señal y atenúa el resto. Luego de aplica el filtro pasa banda,  se cuenta con las frecuencias de interés y se continuá con el procesamiento. Una alternativa es calcular la Densidad Espectral de Potencia (DEP). Esta se define como:

$$ P = \int_{-\inf}^{+inf} S_{xx} (f) df \qquad  \textrm{donde}$$

$$ S_{xx} = |X(f)|^{2} \qquad \textbf{y} \qquad X(f) \textrm{ es la Transformada de \emph{Fourier}} $$

Esta potencia puede ser utilizada luego como una característica de interés. Se discutirá más adelante. Otra alternativa  es, por ejemplo, calcular el promedio de las frecuencias. Las posibilidades aquí son muchas y dependen de lo que se esté buscando.

\section{Extracción de Características de Interés y Clasificación}

\section{EEG}

\section{EMG}

\section{EKG}

\chapter{Implementación}
\section{Hardware}

\section{Procesamiento de Señales}

\section{Desarrollo de universos 3D interactivos}

\addcontentsline{toc}{chapter}{Bibliografía}
\bibliography{bibliography}{}
\bibliographystyle{babplain}
 
\end{document}