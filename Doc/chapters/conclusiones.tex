En este capítulo, se presentaran breves conclusiones sobre el proyecto realizado. 

En el caso de EEG se logró darle un uso que mejora la interacción del usuario con el universo. El problema de yace en que como las ondas \emph{Alfa} varían tanto de persona a persona, hay en personas que no funciona tan bien. Además, la precisión alcanzada, si bien es suficiente para el uso que se le dio, si se le quieren dar otros usos se debe mejorar. A su vez, el usuario debe quedarse muy quieto ya que los movimientos pueden causar que los electrodos generen lecturas incorrectas. Por este motivo, se concluye que la utilización de un EEG en un universo interactivo es realizable pero se deben obtener sensores que sean menos sensibles al movimiento.

El uso del dispositivo EMG fue sin dudas el que mejores resultados arrojó. Es muy fácil de colocar y su precisión fue muy alta. Además, con tan solo unos minutos de entrenamiento ya es suficiente para obtener buenas predicciones.

En el caso del dispositivo SpO\textsubscript{2}, es totalmente utilizable en tiempo real ya que la latencia es mínima y no requiere entrenamiento. El usuario simplemente se lo coloca y puede interactuar con el universo.

Por lo tanto, se concluye que posible procesar bioseñales en tiempo real y utilizar la información en universos interactivos. La pregunta difícil aquí es ¿cómo hacerlo?. Se debe hacer un análisis exhaustivo del estado que se quiera detectar para determinar que nivel mínimo de precisión se requiere y cuál sería el mejor uso dentro del universo. Además, el procesamiento debe ser lo suficientemente rápido para reaccionar rápidamente ante los cambios y no introducir latencia en el universo interactivo.