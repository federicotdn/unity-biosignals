En este capítulo, se presentaran breves conclusiones sobre el proyecto realizado y también se discutirán posibles mejoras y pasos a seguir.

\section{Conclusiones Generales}

En el caso de \acrshort{eeg} se logró darle un uso que mejora la interacción del usuario con el universo. El problema yace en que como las ondas \emph{Alfa} varían en grandes cantidades de persona a persona, hay en personas que no funciona tan bien. Además, la precisión alcanzada, si bien es suficiente para el uso que se le dio, si se le quieren dar otros usos se debería mejorar. A su vez, el usuario debe quedarse muy quieto ya que los movimientos pueden causar que los electrodos generen lecturas incorrectas. Por estos motivos, se concluye que la utilización de un señales \acrshort{eeg} en un universo interactivo es realizable pero no será fácil. Es posible que existan casos de personas en los que no funcionará correctamente ya que el cerebro es muy complejo. Además, el campo \acrshort{bci} se encuentra poco explorado.

El uso del dispositivo \acrshort{emg} fue sin dudas el que mejores resultados arrojó. Es muy fácil de colocar y su precisión fue muy alta. Además, con tan solo unos minutos de entrenamiento ya es suficiente para obtener buenas predicciones. e podrían utilizar dispositivos más ergonómicos que el utilizado para que el usuario se encuentre más confortable.

En el caso del dispositivo \acrshort{spo2}, es totalmente utilizable en tiempo real ya que la latencia es mínima y no requiere entrenamiento. El usuario simplemente se lo coloca y puede interactuar con el universo. En este caso la dificultas más grande es crear universos lo suficientemente reales e inmersivos para lograr grandes variaciones en el ritmo cardíaco. El problema que se encontró en el dispositivo utilizado fue, que al utilizarse en un dedo, resultaba incómodo el uso del ratón. Se podría utilizar un dispositivo más ergonómico en la muñeca por ejemplo.

Por lo tanto, se concluye que posible procesar bioseñales en tiempo real y utilizar la información en universos interactivos. La pregunta difícil aquí es ¿cómo hacerlo?. Se debe hacer un análisis exhaustivo del estado que se quiera detectar para determinar que nivel mínimo de precisión se requiere y cuál sería el mejor uso dentro del universo. Además, el procesamiento debe ser lo suficientemente rápido para reaccionar rápidamente ante los cambios y no introducir latencia en el universo interactivo.

\section{Mejoras Posibles y Próximos Pasos}

Se podría, por ejemplo, medir el nivel de fuerza ejercido utilizando un dispositivo \acrshort{emg} en lugar de que sea una medición binaria. El uso de bioseñales resulta mucho más interesante en el campo de la realidad virtual debido a que el usuario se encuentra totalmente inmerso en un universo. Aquí, cuanta mayor información del usuario se pueda obtener, mejor será la experiencia. Aparecerían otras limitaciones como el poder de procesamiento, ya que la realidad virtual demanda demasiado del mismo, o, el hecho de que el usuario se encuentra con un casco en la cabeza. Yendo aún más lejos, usando un dispositivo \acrshort{eeg}, se podrían utilizar las ondas relacionadas a la corteza motora para detectar la voluntad de mover un brazo y simular dicho movimiento en un universo en realidad virtual. De esta forma, una persona que por alguna incapacidad no puede mover un brazo, lo pueda utilizar.

Este trabajo se podría utilizar como base para trabajos futuros. Se podrían utilizar las librerías utilizadas para procesar las señales de otra forma y reconocer aún más estados del usuario. También se podría adaptar el trabajo para que funcione con un dispositivo de realidad virtual. En fin, las aplicaciones del uso de bioseñales son muchas y se pueden lograr grandes cosas. A medida que vayan surgiendo más y más universos interactivos que utilicen bioseñales, más tracción cobrará el campo y más se podrá avanzar. Este trabajo es un aporte a este campo.