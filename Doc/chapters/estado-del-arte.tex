En este capítulo se presentará el estado del arte en lo que respecta a los distintos usos de las bioseñales. Primero se mostrará el uso de las bioseñales en Accesibilidad, luego su uso en aplicaciones interactivas y finalmente se sacarán conclusiones. 

\section{Uso de bioseñales en distintos campos}

La computadora utilizada por Stephen Hawking es tal vez el caso más conocido de la utilización de bioseñales en Accesibilidad. Stephen Hawking cuenta con esclerosis lateral amiotrófica, por lo que se encuentra paralizado y no puede hablar. Para poder comunicarse, Intel desarrolló un sistema compuesto por una tableta y un sensor infrarojo montado en sobre sus anteojos. El sensor infrarojo detecta el movimiento en su cachete izquierdo. La tableta cuenta con una plataforma de código abierto llamada ACAT. ACAT provee un teclado virtual en la pantalla. Utilizando el movimiento de su cachete, Hawking, puede detener el cursor donde desea y así, escribir. Es decir, es una entrada binaria. Este también utiliza un procesador de texto con predicción de palabras que permite acelerar el proceso.  Luego, el sistema utiliza un sintetizador de voz para comunicar lo que escribió. Esta, es solo una de las aplicaciones de ACAT. ACAT también le permite controlar el ratón en \emph{Windows}, y así, controlar completamente la computadora para poder utilizar su correo electrónico, navegar por internet, entre otras cosas \cite{hawking}.


\emph{Netflix} desarrolló \emph{MindFlix}. \emph{Mindflix} utiliza un dispositivo EEG (electroencefalograma) para controlar su popular servicio con la mente. Utiliza los giroscopios del dispositivo para permitirle al usuario desplazarse horizontalmente y verticalmente por la interfaz. Además, utiliza distintas ondas cerebrales para detectar cuando el usuario piensa en la palabra \emph{play}. En caso de que el usuario piense en esa palabra,  la aplicación comienza a reproducir el contenido seleccionado. Se intentó averiguar qué ondas cerebrales se utilizaban y de que forma no se encontró en ningun lugar \cite{mindflix}.

El \emph{Star Wars Force Trainer} es un juguete que utiliza un EEG para medir las ondas cerebrales \emph{beta}. Utilizando dichas ondas, determina que tan concentrado esta el usuario y le permite "controlar con la mente" una pelota de ping pong. El problema de utilizar estas ondas es que pueden activarse por otros motivos y no solo por la concentración \cite{force}.


\section{Conclusiones}

Las aplicaciones de las bioseñales son infinitas y cada vez hay más interés en su uso. Pueden utilizarse tanto para ayudar a personas con discapacidad como también para entretener. Todas las implementaciones combinan \emph{harware} y \emph{sowftware} pero algunas más que otras. En el caso de la computadora de Stephen Hawking, si bien utiliza un sensor infrarojo, el logro aquí está en el software que utiliza ya que el sensor no es tan sofisticado. En los casos de \emph{MindFlix} y \emph{Force Trainer}, el harware cobra más importancia, ya que las ondas del cerebro son difíciles de leer y se debe obtener el menor ruido posible.