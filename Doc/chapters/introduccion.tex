Este trabajo tuvo como objetivo el procesamiento de bioseñales en tiempo real en universos interactivos. Una bioseñal puede ser definida como una descripción de un fenómeno fisiológico \cite{biosignal-book-2}.  El término ``tiempo real'', aquí, se refiere a computación gráfica en tiempo real, es decir, la generación de imágenes lo suficientemente rápido como para crear la ilusión de movimiento. Los universos interactivos son simulaciones que obtienen entradas del usuario y reaccionan antes las mismas. Las entradas más comunes son el ratón y el teclado. En este trabajo se investigó la utilización de bioseñales como entradas para augmentar la inmersión.

Hasta la fecha no se ha encontrado ningún uso comercial de bioseñales en universos interactivos. Si se han encontrado aplicaciones realizadas en centros de investigación o universidades, los cuáles se detallarán en la sección \ref{chap:biosignal-apps}. Se buscó determinar si determinadas bioseñales eran aptas para la utilización en tiempo real en universos interactivos y cómo su utilización mejoran la experiencia de usuario. El objetivo de este proyecto no fue ofrecer un análisis cualitativo de los diferentes métodos de procesamiento de señales, sino cómo utilizar los existentes para mejorar la inmersión en universos interactivos teniendo en cuenta los desafíos que esto implica de de 