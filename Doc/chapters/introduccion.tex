Este trabajo tuvo como objetivo el procesamiento de bioseñales en tiempo real en universos interactivos. Una bioseñal puede ser definida como una descripción de un fenómeno fisiológico \cite{biosignal-book-2}.  El término ``tiempo real'', aquí, se refiere a computación gráfica en tiempo real, es decir, la generación de imágenes lo suficientemente rápido como para crear la ilusión de movimiento. Los universos interactivos son simulaciones con componentes visuales que obtienen entradas del usuario y reaccionan antes las mismas. Las entradas más comunes son el \emph{mouse} y el teclado. En este trabajo se investigó la utilización de bioseñales como entradas adicionales para aumentar la inmersión.

Se han encontrado muy pocas aplicaciones de uso comercial de bioseñales en universos interactivos y su alcance es muy reducido\cite{neurogaming}. Se buscó determinar si \acrshort{eeg}, \acrshort{emg} y \acrshort{spo2} eran aptos para la utilización en tiempo real en universos interactivos y cómo su utilización podría mejorar la experiencia de usuario, contando sólo con dispositivos que puedan ser adquiridos en el mercado, y no dispositivos profesionales de alto costo. Los dispositivos y señales fueron elegidos a partir de cuatro criterios que se detallan más adelante. El objetivo de este proyecto no fue ofrecer un análisis cualitativo y cuantitativo de los diferentes métodos de procesamiento de señales, sino cómo utilizar los existentes para posiblemente mejorar la inmersión en universos interactivos teniendo en cuenta los desafíos que esto implica. Este trabajo tampoco buscó determinar si mejoraba o no la inmersión, sino que se limitó a verificar si era posible encontrar un esquema donde se utilizaran bioseñales en tiempo real en universos interactivos.